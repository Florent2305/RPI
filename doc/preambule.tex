\documentclass[11pt,a4paper,twoside,openright]{book}

% Standard packages
\usepackage{amsmath, amsthm, amssymb, graphicx, xcolor, listings, dsfont}
\usepackage[utf8]{inputenc}
%\usepackage{courier}

% PDF link
% \usepackage[pagebackref]{hyperref}
% \hypersetup{
%    bookmarks=true,					% show bookmarks bar?
%    unicode=false,					% non-Latin characters in Acrobat's bookmarks
%    pdftoolbar=true,					% show Acrobat's toolbar?
%    pdfmenubar=true,					% show Acrobat's menu?
%    pdffitwindow=true,				% page fit to window when opened
%    pdftitle={Registration},	    	% title
%    pdfauthor={Vincent Garcia},		% author
%    pdfsubject={},	                % subject of the document
%    pdfnewwindow=false,				% links in new window
%    pdfkeywords={},					% list of keywords
%    colorlinks=true,					% false: boxed links; true: colored links
%    linkcolor=red,					% color of internal links
%    citecolor=green,					% color of links to bibliography
%    filecolor=magenta,				% color of file links
%    urlcolor=cyan					% color of external links
% }


% Paths to figures
\graphicspath{{./figures/}}


% Colors
\definecolor{codelightgray}{rgb}{0.5,0.5,0.5}
\definecolor{keywords}{RGB}{128,128,0}

\definecolor{comments}{RGB}{0,128,0}


% Listings
%\lstset{
%language=C++,
%frame=lines,
%basicstyle=\footnotesize\ttfamily\color{codelightgray},
%xleftmargin=0.15cm,
%xrightmargin=0.15cm,
%keywordstyle=\color{keywords},
%commentstyle=\color{comments}\emph,
%captionstyle=\color{comments}\small
%}

%listings parameters


\definecolor{light-gray}{gray}{0.95}

\lstset{ %
language=C++, % choose the language of the code
basicstyle=\small, % the size of the fonts that are used for the code
%numbers=left, % where to put the line-numbers
%numberstyle=\tiny, % the size of the fonts that are used for the line-numbers
%stepnumber=2, % the step between two line-numbers. If it's 1 each line
%% will be numbered
%numbersep=5pt, % how far the line-numbers are from the code
backgroundcolor=\color{light-gray}, % choose the background color. You must add \usepackage{color}
showspaces=false, % show spaces adding particular underscores
showstringspaces=false, % underline spaces within strings
showtabs=false, % show tabs within strings adding particular underscores
frame=single, % adds a frame around the code
tabsize=2, % sets default tabsize to 2 spaces
captionpos=t, % sets the caption-position to bottom
breaklines=true, % sets automatic line breaking
breakatwhitespace=false, % sets if automatic breaks should only happen at whitespace
title=\lstname, % show the filename of files included with \lstinputlisting;
% also try caption instead of title
escapeinside={\%*}{*)}, % if you want to add a comment within your code
morekeywords={*,slots,signals,connect,SIGNAL,SLOT}, % if you want to add more keywords to the set
keywordstyle=\color{keywords},
commentstyle=\color{comments}\emph%,captionstyle=\color{comments}\small
}



% \lstset{ %
% language=C++,
% basicstyle=\footnotesize,       % the size of the fonts that are used for the code
% backgroundcolor=\color{white},  % choose the background color. You must add \usepackage{color}
% showspaces=false,               % show spaces adding particular underscores
% showstringspaces=false,         % underline spaces within strings
% showtabs=false,                 % show tabs within strings adding particular underscores
% frame=single,                   % adds a frame around the code
% tabsize=2,                      % sets default tabsize to 2 spaces
% captionpos=b,                   % sets the caption-position to bottom
% breaklines=true,                % sets automatic line breaking
% breakatwhitespace=false,        % sets if automatic breaks should only happen at whitespace
% title=\lstname,                 % show the filename of files included with \lstinputlisting;
%                                 % also try caption instead of title
% escapeinside={\%*}{*)},         % if you want to add a comment within your code
% morekeywords={*,...}            % if you want to add more keywords to the set
% }
